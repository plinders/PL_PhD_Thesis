%!TEX root = ../dissertation.tex
% Insert PDF
% ex. \includepdf{figures/inserts/ch6.pdf}
\includepdf{figures/inserts/ch8.pdf}

% Set chapter number color to be similar to the insert color
\definecolor{chaptergrey}{rgb}{0.203063, 0.379716, 0.553925}

\chapter{Addendum}

\clearpage
\addthumb{\thechapter}{\Large{\thechapter}}{white}{chaptergrey}

\section{Nederlandse Samenvatting}



\clearpage

\section{Dankwoord}



\clearpage

\section{Curriculum Vitae (English)}

Peter Linders was born on April 28\textsuperscript{th} 1993 in Oss, the Netherlands. After completing his secondary education in 2011, he started his Bachelor of Sciences degree in Biomedical Sciences at the Radboud University in Nijmegen, the Netherlands. For his Bachelor's internship in 2014, he joined the group of dr. Egbert Oosterwijk at the Department of Experimental Urology at the Radboud Institute for Molecular Life Sciences. Supervised by dr. Marije Sloff and dr. Silvia Mihaila, he investigated the use of adipose-derived stem cells for human urogenital tissue engineering. In 2014, Peter obtained his BSc degree majoring in Human Pathobiology.

Peter continued his education by pursuing a Master of Sciences degree in Biomedical Sciences at the Radboud University with a major in Human Pathobiology. During his first internship in early 2015, he joined the group of prof. dr. Geert van den Bogaart at the Department of Tumor Immunology at the Radboud Institute for Molecular Life Sciences. Under the supervision of Geert and dr. Ilse Dingjan, he worked on the recruitment of NADPH oxidase 2 in dendritic cells. For his second internship in late 2015 and early 2016, he moved to Basel, Switzerland to join the Novartis Institutes for BioMedical Research. Supervised by dr. Emilie Chapeau and dr. Ralph Tiedt, he worked on large-scale experiments to identify new genes involved in cytotoxic T cell activation and exhaustion, using the \emph{piggyBac} transposon system. In 2016, Peter returned to Nijmegen and obtained his MSc degree \emph{bene meritum}.

Using his background in both cell biology and microscopy, Peter started his PhD training in the Membrane Trafficking group of the Tumor Immunology at the Radboud Institute for Molecular Life Sciences in Nijmegen, the Netherlands, supervised by prof. dr. Geert van den Bogaart and prof. dr. Dirk Lefeber. As described in this thesis, he worked on understanding the molecular mechanisms in rare glycosylation disorders. Peter has presented his work at various international conferences and has published several papers in peer-reviewed international scientific journals. During his PhD, Peter has also been an eLife Community Ambassador and was chair of the Radboud Consortium for Glycoscience PhD Council. 

\clearpage

\section{Curriculum Vitae (Nederlands)}

Peter Linders werd geboren op 28 april 1993 te Oss. Na het behalen van zijn vwo-diploma in 2011, begon hij met de bachelor Biomedische Wetenschappen aan de Radboud Universiteit Nijmegen. Voor zijn bachelor stage in 2014 sloot hij zich aan bij de onderzoeksgroep van dr. Egbert Oosterwijk bij de afdeling Experimentele Urologie in het Radboud Institute for Molecular Life Sciences. Onder begeleiding van dr. Marije Sloff en dr. Silvia Mihaila, werkte hij aan het gebruik van vetstamcellen in tissue engineering van menselijk urogenitaal weefsel. In 2014, ontving hij zijn BSc diploma met als hoofdvak Humane Pathobiologie.

Peter vervolgde zijn studies door te beginnen aan de master Biomedical Sciences aan de Radboud Universiteit, met een specialisatie in Human Pathobiology. Tijdens zijn eerste masterstage in begin 2015, werd hij deel van de onderzoeksgroep van prof. dr. Geert van den Bogaart bij de afdeling Tumorimmunologie in het Radboud Institute for Molecular Life Sciences. Onder begeleiding van Geert en dr. Ilse Dingjan onderzocht hij het transport van NADPH oxidase 2 naar het fagosoom in dendritische cellen. Voor zijn tweede masterstage eind 2015 tot en met begin 2016 verhuisde hij naar de Novartis Institutes for BioMedical Research in Basel, Zwitserland. Bij de farmacologieafdeling, begeleidt door dr. Emilie Chapeau en dr. Ralph Tiedt, werkte hij aan grootschalige experimenten om nieuwe genen te ontdekken die belangrijk zijn voor cytotoxische T-cel activatie en uitputting, door middel van het \emph{piggyBac} transposon systeem. In 2016 keerde Peter terug naar Nijmegen en ontving hij zijn MSc diploma met het judicium \emph{bene meritum}.

Zijn achtergrond in zowel celbiologie als microscopie toepassend, startte Peter met zijn promotietraject in de Membrane Trafficking onderzoeksgroep van de afdeling Tumorimmunologie bij het Radboud Institute for Molecular Life Sciences, met prof. dr. Geert van den Bogaart en prof. dr. Dirk Lefeber als promotoren. Zoals beschreven in dit proefschrift werkte hij aan het begrijpen van de moleculaire mechanismen in zeldzame glycosyleringsafwijkingen. Peter heeft zijn werk bij verscheidende internationale conferenties gepresenteerd en heeft meerdere publicaties in internationale, peer-reviewed wetenschappelijke tijdschriften. Naast zijn promotieonderzoek was Peter ook betrokken als eLife Community Ambassador en tevens voorzitter van het Radboud Consortium for Glycoscience PhD Council.


\clearpage

\section{List of Publications}

\textbf{Linders P.T.A.}, Peters E., ter Beest M., Lefeber D.J., and van den Bogaart G. \\
\textbf{Sugary Logistics Gone Wrong: Membrane Trafficking and Congenital Disorders of Glycosylation} \\
\emph{International Journal of Molecular Sciences 21 p. 4654, 2020}

\vspace{\baselineskip}

\noindent\textbf{Linders P.T.A.}, Gerretsen E., Ashikov A., Vals M.-A., Revelo N.H., Arts R., Baerenfaenger M., Zijlstra F., Huijben K., Raymond K., Muru K., Fjodorova O., Pajusalu S., Õunap K., ter Beest M., Lefeber D.J., and van den Bogaart G. \\
\textbf{Congenital disorder of glycosylation caused by starting site-specific variant in syntaxin-5} \\
\emph{Manuscript in submission at Nature Communications, 2020}

\vspace{\baselineskip}

\noindent\textbf{Linders P.T.A.}, van der Horst C., ter Beest M. and van den Bogaart G. \\
\textbf{Stx5-Mediated ER-Golgi Transport in Mammals and Yeast} \\
\emph{Cells 8 p. 780, 2019}

\vspace{\baselineskip}

\noindent Paardekooper L.M., Dingjan I., \textbf{Linders P.T.A.}, Staal A.H.J., Cristescu S.M., Verberk W.C.E.P. and van den Bogaart G. \\
\textbf{Human Monocyte-Derived Dendritic Cells Produce Millimolar Concentrations of ROS in Phagosomes Per Second} \\
\emph{Frontiers in Immunology 10 p. 1216, 2019}

\vspace{\baselineskip}

\noindent Ashikov A., Nurulamin A.B., Xiao-Yan W., Niemeijer M., Osorio G.R.P., Brand-Arzamendi K., Hasadsri L., Hansikova H., Raymond K., Simon D.V.M.E.H., Pfundt R., Timal S., Beumers R., Biot C., Smeets R., Kersten M., Huijben K., CDG group, \textbf{Linders P.T.A.}, van den Bogaart G., van Hijum S.A.F.T., Rodenburg R., van den Heuvel L.P., van Spronsen F., Honzik T., Foulquier F., van Scherpenzeel M. and Lefeber D.J. \\
\textbf{Integrating glycomics and genomics uncovers SLC10A7 as essential factor for bone mineralization by regulating post-Golgi protein transport and glycosylation} \\
\emph{Human Molecular Genetics 27(17) p. 3029, 2018}

\vspace{\baselineskip}

\noindent Dingjan I., \textbf{Linders P.T.A.}, Verboogen D.R.J., Revelo N.H., ter Beest M. and van den Bogaart G. \\
\textbf{Endosomal and Phagosomal SNAREs} \\
\emph{Physiological Reviews 98(3) p. 1465, 2018}

\vspace{\baselineskip}

\noindent Dingjan I.,  \textbf{Linders P.T.A.}, van den Bekerom L., Baranov M.V., Halder P., ter Beest M. and van den Bogaart G. \\
\textbf{Oxidized phagosomal NOX2 complex is replenished from lysosomes} \\
\emph{Journal of Cell Science 130 p. 1285, 2017}

