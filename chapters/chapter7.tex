%!TEX root = ../dissertation.tex
% Insert PDF
% ex. \includepdf{figures/inserts/ch6.pdf}
\includepdf{figures/inserts/ch7.pdf}

% Set chapter number color to be similar to the insert color
\definecolor{chaptergrey}{rgb}{0.179019, 0.433756, 0.55743}
% Set citation number color to be similar to the insert color
\definecolor{SchoolColor}{rgb}{0.179019, 0.433756, 0.55743}

\chapter{Discussion \& Future Perspectives}
\addthumb{\thechapter}{\Large{\thechapter}}{white}{chaptergrey}
\stopthumb

\clearpage

\continuethumb
\section{The Clinical Foundation of this Thesis}

The field of membrane trafficking in glycobiology is rapidly advancing. The work presented in this thesis represents only a fraction of the new developments of the past five years. While much is still unknown, many advancements have contributed to the formation of this thesis. The starting point for much of the experimental work presented in this thesis was found in the clinic: chapters 4 and 6 originate from novel genetic variants in the genes coding for the membrane trafficking-related proteins TMEM199 and syntaxin-5 observed in patients with congenital disorders of glycosylation (CDG). Recent advances in the diagnostics of CDG have not only improved the sensitivity of detection of various glycosylation defects but also provide insights as to where exactly in the glycosylation process the defects occur. Where classic diagnostic approaches only interrogate the sialic acid content of glycans using isoelectric focusing (IEF)\cite{marklova_screening_2007}, modern approaches use mass spectrometry to investigate full glycan structure profiles in an unbiased fashion\cite{abu_bakar_clinical_2018,van_scherpenzeel_high-resolution_2015,hipgrave_ederveen_dissecting_2020}. For Stx5-CDG, discussed in chapter 6, the combination of IEF and mass spectrometry-based diagnostics allowed me to identify a glycan profile unique to Stx5-CDG patients (strong decrease of sialic acid and galactose incorporation, but strong increase in Man-5 glycan abundance). Moreover, this observation can already be linked to a step in the glycosylation pathway: as the Man-5 glycan is a very early N-glycan, the defect must relate to a component early in the Golgi apparatus. This information would have been overlooked if only the IEF data would have been considered. Together with advances in next-generation sequencing\cite{de_ligt_diagnostic_2012,gilissen_genome_2014}, the genomic location of the variant could quickly be identified in STX5. This offered a stepping stone to understanding the cell biological mechanism behind the clinical presentation. Human disease therefore can bring an understanding of the function of proteins in the cell, and chapters 4 and 6 are good examples of how advancements in CDG diagnostics pave the way to elucidate the role of membrane trafficking components.

Inversely, understanding the cell biological mechanisms in disease is paramount for the development of treatment options. As CDGs are rare diseases that do not affect many individuals\cite{linders_sugary_2020}, treatment options are scarce but desirable. A better understanding of the underlying cell biological and biochemical processes affected in the cells of CDG patients opens up new therapeutic opportunities, which can be as straightforward as monosaccharide supplementation\cite{witters_d-galactose_2021,wong_oral_2017,voermans_pgm1_2017}. In some cases, such as the Stx5-CDG patients presented in chapter 6, treatment options are currently not feasible. Considering the breadth of the disorder, and that it likely affects embryonic development, straightforward supplementation strategies would not be sufficient to ameliorate the patient phenotypes. Possible therapeutic opportunities include the use of gene replacement therapy, similar to the treatment of spinal muscular atrophy (SMA)\cite{mendell_single-dose_2017}. For Stx5-CDG patients, this would need to be performed early in development and is therefore technologically and ethically complicated.

These rare diseases also offer the opportunity for new initiatives both for patients and researchers. One such opportunity is the Noordzeeziekte foundation\cite{noauthor_home_nodate}, which is a patient organization for individuals suffering from the homozygous G144W mutation in the Golgi trafficking protein GOSR2\cite{praschberger_mutations_2017,praschberger_expanding_2015,corbett_mutation_2011}, partner SNARE for syntaxin-5. Noordzeeziekte has been identified in 25 patients so far worldwide, with most of them living in the north of the Netherlands, and is characterized by muscle spasms and epilepsy. The Noordzeeziekte foundation offers a platform for (family of) patients to understand the disease and the therapeutic options, as well as to share the research progress of scientists affiliated with the foundation. Similarly, the foundation enables the continuation of basic research through donations, advancing the understanding and treatment options of the disease. Analogous patient organizations also exist for CDG in many different countries\cite{noauthor_cdg_nodate}.

\section{On Tissue-Specific Glycosylation}

Many CDGs are characterized by tissue-specific defects, even though the affected machinery is present in most cells of the human body. In this thesis, I discuss Stx5M55V-CDG (chapter 6) which primarily has skeletal and hepatological involvement, TMEM199-CDG (chapter 4) which mainly affects the liver\cite{jansen_tmem199_2016}, and CCDC115-CDG\cite{jansen_ccdc115_2016} which affects the liver, spleen and, brain. Especially TMEM199 and CCDC115 are of interest as it is hypothesized that they act in concert to facilitate V-ATPase assembly\cite{miles_vacuolar-atpase_2017}, and my data in chapter 4 suggests that TMEM199 recruits CCDC115 to membranes, raising the question why the patient phenotype is different. Another clear example of tissue-specific glycosylation is found in GNE-CDG, caused by genetic variants in GNE (Glucosamine (UDP-N-Acetyl)-2-Epimerase/N-Acetylmannosamine Kinase), the enzyme responsible for the two first steps in sialic acid biosynthesis\cite{argov_rimmed_1984,broccolini_hereditary_2011,brasil_cdg_2018,voermans_clinical_2010,willems_genetic_2016}. In GNE-CDG, all skeletal muscle tissue is affected except for the quadriceps. A likely explanation for this phenomenon is that the expression levels of glycosyltransferases vary between tissues and cell types\cite{linders_sugary_2020,papanikou_yeast_2009,mogelsvang_predicting_2004,nilsson_overlapping_1993,ripoche_location_1994,velasco_cell_1993,nairn_regulation_2008}. Several distinct isoforms exist of most glycosyltransferases\cite{moremen_vertebrate_2012}, which is likely important for the tissue distribution of glycans. Moreover, differences in expression levels of glycosyltransferases can shift the balance towards distinct glycosylation steps, similarly how the ratio of syntaxin-5 isoform expression can dictate intracellular trafficking as seen in chapter 6. The ratio of glycosylation enzymes might hence alter both the substrate and product levels of certain intermediate glycans, producing different glycans on the same protein in different cell types.

The studies I performed in chapters 3, 4, and 6 in this thesis made use of one or two different cell types, but it would be much more valuable to interrogate glycosylation disorders caused by mutant trafficking proteins in various types of tissues. induced pluripotent stem cells (iPSCs) are ideal with their capacity to differentiate in multiple tissues. Coupled with the retention using selective hooks (RUSH) system\cite{boncompain_synchronization_2012} and glycoproteomics, glycoproteins can be followed almost in real-time during their transit through the secretory pathway in multiple cell types, with concurrent analysis of the attached glycan structures. Proteins of interest fused to fluorescent proteins can be imaged by live-cell microscopy, while the same tag can be used to purify the glycoprotein via immunoprecipitation for downstream mass spectrometry analysis of not only glycans but also interacting proteins at each step in the trafficking pathway. In theory, this system could also be multiplexed by using multiple different cargo molecules (e.g., cytokines, ECM proteins, or immunoglobulins) with different (fluorescent) tags. Using such high spatial and temporal resolution where glycosylation and trafficking meet is highly interesting to elucidate how and where differential glycosylation occurs. Moreover, the same processes can be studied in iPSCs derived from CDG patients to observe how mutations affect the transit of glycoproteins.

\section{The Use of Primary Human Material in Cell Biological Studies}

One important limitation from chapter 6 concerns the use of primary human dermal fibroblasts. While primary patient material is the most accurate model to study disease effects on a cellular level, there are three important aspects to using patient-derived cells: (1) high interindividual variation, (2) primary patient material is mostly difficult to manipulate by i.e., fluorescent protein expression, and (3) primary human dermal fibroblasts are often not the most relevant cell type with regards to the pathophysiology. 

First, as all individuals are unique, so are the cells derived from both healthy donors and patients. As opposed to studies performed in genetically uniform cell lines or animals, many considerations need to be taken when working with material from humans. Many different aspects including age, treatment, diet, environment, and genetic makeup influence the biology of cells in the human body and therefore influence the results of cell-based assays. We observed this by different protein expression levels between donors as measured by Western blot, as well as by the different glycosylation results dependent on the availability of N-glycans in blood serum. While this affects the interpretation of these experiments, an important improvement would be to include more different patients to account for the variation. Ultimately, the observations in chapter 6 are a testament to how humans are all unique.

Second, cell biological techniques often use the exogenous expression of fluorescent proteins in live cells to interrogate many different processes. For short-term experiments, a suitable technique involves transient delivery, or transfection, of plasmid DNA for temporary expression of the desired protein. Many different methods for transient transfection exist, such as complexing plasmid DNA with cationic lipids for endosomal uptake\cite{luo_synthetic_2000,lonez_cationic_2008} or high voltage electroporation\cite{luo_directing_2019}, but these are often not ideal for the transfection of primary material compared to transformed and immortalized cell lines\cite{yamano_comparison_2010,maurisse_comparative_2010}. This severely limits the options for studying cell biological processes in live primary cells. Viral delivery of plasmid DNA is generally a more suitable option for stable transduction of primary human cells and offers high levels of transduction efficiency. Unfortunately, viral transduction methods come with severe drawbacks regarding the production of viral particles, toxicity, and immunogenicity which are undesirable in certain experimental contexts\cite{luo_synthetic_2000,anderson_human_1998,crystal_transfer_1995,tripathy_immune_1996}.

The most ideal method would be to generate cell lines that express fluorescent (fusion) proteins in a mostly endogenous context. CRISPR/Cas9 technology has enabled robust genomic engineering with a relatively small investment necessary, enabling wide-spread applications\cite{hsu_development_2014,cho_targeted_2013,cong_multiplex_2013,jinek_programmable_2012}. Recently, improvements to the CRISPR/Cas9 system facilitate the development of genome-edited cells even further by increased specificity and less off-target effects\cite{slaymaker_rationally_2016} and by inferring resistance to a potent cytotoxic drug concurrently with the desired edit to improve the selection of positive cells\cite{agudelo_marker-free_2017}. Most importantly, well-designed CRISPR/Cas9 constructs do not affect the nucleotide context of the edited gene, therefore maintaining physiological promoter function and expression regulation. This eliminates artifacts commonly associated with overexpression\cite{ratz_crisprcas9-mediated_2015,ali_optimizing_2018}. Experiments of use would be to perform the same FLIM-FRET-based approach as outlined in chapter 6 but with endogenous SNAREs, to truly measure SNARE fusion \emph{in situ} in an endogenous system. The same is true for other trafficking assays; when the RUSH system is introduced endogenously rather than with overexpression, all copies of the protein of interest can be captured, thereby removing the effect of the unedited protein from the experiment. Another noteworthy avenue would be to perform CRISPR/Cas9 genome editing in iPSCs\cite{klimanskaya_human_2006,takahashi_induction_2006}. The pluripotent property of these cells would allow the investigation of protein function in many different tissue types, all arising from a single progenitor. Moreover, robust iPSC generation can be performed using from patient-derived fibroblasts from patients with uncharacterized mutations resulting in, for example, CDGs. 

Another interesting addition to the genome editing arsenal is the development of base editing and prime editing\cite{komor_programmable_2016,huang_precision_2021,anzalone_search-and-replace_2019}. Base and prime editing use a similar genomic targeting strategy to CRISPR/Cas9 and allow for precise single nucleotide changes. New generation prime editing tools use a catalytically impaired Cas9 fused to reverse transcriptase\cite{anzalone_search-and-replace_2019}. A prime editing guide RNA (pegRNA) specifies both the target site for the fusion complex and the desired genomic edit, which is then inserted using a single strand DNA break. Contrary to CRISPR/Cas9, base and prime editing do not induce double-strand breaks in DNA which are the source of random insertions and deletions at the target locus\cite{cox_therapeutic_2015,hilton_enabling_2015}. Moreover, double-strand breaks can induce a p53-mediated DNA damage response which strongly inhibits the application of CRISPR/Cas9 in pluripotent stem cells\cite{ihry_p53_2018,haapaniemi_crisprcas9_2018}.

While prime and base editing can only edit a single nucleotide, the technology is of major interest to deploy in patient cells. In our case, we could have repaired the single nucleotide mutation in Stx5M55V patient cells (chapter 6) to investigate whether restoring Stx5S protein expression would rescue the patient phenotype. These technologies are currently only feasible in laboratory settings, however, in the future, these strategies could also be applied to efficiently and safely restore congenital mutations in humans.

\section{How Advancements in Microscopy Advance Golgi and Glycosylation Research}

Other advancements that are of interest regarding this thesis are advances in optical microscopy. Most optical microscopy presented in this thesis was performed using confocal microscopy systems, which have a maximum resolution of 200 nm in the \emph{x} and \emph{y} dimensions and about 1 $\mu$m in the \emph{z} dimension (depending on the wavelength of the emitted light)\cite{st_croix_confocal_2005}. A major drawback is that the space between cisterna, approximately 20 nm, is indiscernible using confocal microscopy, and the highly compact nature of the Golgi makes it very difficult to resolve fluorescent signals belonging to different Golgi cisternae\cite{ladinsky_golgi_1999}. This is of particular importance for chapters 4 and 6, where I used confocal microscopy to investigate the localization of glycosyltransferases in Golgi cisternae. As confocal microscopy lacks the resolution to accurately separate the different cisternae, I could only make general conclusions about the localization of the glycosyltransferases instead of accurately naming the cisternae where the protein of interest resides in. The latter would be of much interest for understanding glycosylation in the context of different disorders. Using computational approaches, it is currently possible to resolve cargo in transit between different cisternae\cite{dickson_rab6_2020} but this is still relatively complicated and does not offer a high temporal resolution.

Recently, super-resolution microscopy techniques such as STED (stimulated emission depletion), SIM (structured illumination microscopy), PALM (photo-activated localization microscopy), and STORM (stochastic optical reconstruction microscopy) have become much more accessible and could therefore be of interest to image Golgi trafficking and related processes. Traditionally, electron microscopy has been applied to visualizing Golgi membranes, but more recent approaches combine both electron microscopy and optical fluorescence microscopy as correlated light and electron microscopy (CLEM)\cite{de_boer_correlated_2015}. CLEM allows maintaining the flexibility of fluorescence-based optical microscopy experiments, potentially with super-resolution modalities\cite{joosten_super-resolution_2018}, with the robust lateral resolution of electron microscopy. Improving the resolution axially and laterally is possible through focused ion beam milling combined with scanning electron microscopy (FIB-SEM) and cryogenic super-resolution to obtain images at 4 nm isotropic resolution with 40 nm accuracy in fluorescence imaging\cite{hoffman_correlative_2020}. While the lateral and axial resolution is unparalleled with FIB-SEM with cryo-SIM, it is highly restricted in temporal resolution as live samples cannot be imaged and full image acquisition per sample can be very long. Moreover, depending on the complexity and dimensions of the sample, datasets can be restrictively large and typically range from 100 GB to about 19.5 TB\cite{hoffman_correlative_2020,xu_enhanced_2017}, making FIB-SEM with cryo-SIM difficult to scale. 

As in many optical microscopy techniques, any increases in spatial resolution decrease the temporal resolution. This raises the question: what is the highest spatial resolution achievable with high temporal resolution? A perfect technique for the time-lapse experiments I performed in this thesis would be lattice light sheet microscopy (LLSM)\cite{chen_lattice_2014}. LLSM can currently achieve the highest spatiotemporal resolution in biological samples with SIM-levels of spatial resolution and temporal resolution of only a few milliseconds\cite{chen_lattice_2014}. The underlying principle is the creation of a structured light sheet that only illuminates a very small volume in the cell (approx. 1 $\mu$m deep) which is then captured by a camera. This enables confocal-like imaging (i.e., imaging a small focal plane rather than the whole sample as with epifluorescence microscopy) with the capture speed of epifluorescence microscopy as images are not reconstructed from single-pixel scans. Furthermore, phototoxicity is significantly reduced by the decrease of sample illumination, thus making LLSM particularly suitable for imaging cells in all dimensions of spacetime. Current LLSM setups are also suitable for fluorescence lifetime imaging microscopy (FLIM)\cite{hirvonen_lightsheet_2020}, a technique I applied to several research questions throughout this thesis (chapters 3, 4, and 6).

LLSM could greatly improve the time-lapse studies I performed in this thesis. For instance, following the transit of tumor necrosis factor (TNF)-$\alpha$ in chapter 3 was currently performed at a temporal resolution of 1 minute per frame. LLSM could speed this imaging up by orders of magnitude, thereby enabling highly accurate pH measurements in time. The same holds for the syntaxin-5 trafficking experiments of chapter 6. With LLSM, the localization of the syntaxin-5 isoforms could much more accurately be determined, both spatially and temporally. LLSM hence allows for much more precise analyses of intracellular trafficking dynamics. As the exact mechanisms of Golgi transport are still unclear, LLSM could offer additional insights into how Golgi-related trafficking processes occur.


\section{Intersecting Membrane Trafficking and Glycosylation}

The work I performed in this thesis contributes to the understanding of intracellular membrane trafficking in the context of glycosylation. Understanding the fundamental transport processes of the Golgi improves our understanding of glycosylation, and vice versa, understanding the exact mechanism of glycosylation improves our understanding of Golgi membrane trafficking. 

Even though the complexity of Golgi transport transcends current advances in biology, a variation of the cisternal maturation model seems the most likely transport theory\cite{glick_membrane_2009,glick_models_2011,pantazopoulou_kinetic_2019}. Cisternal maturation postulates that a new \emph{cis}-Golgi cisterna is formed from vesicles containing newly synthesized (glyco)proteins emanating from the ER. This new cisterna is then matured by renewing its biochemical contents, through donating some of its Golgi-resident proteins to younger cisternae and receiving other Golgi-resident proteins from older cisternae. This is a continuous process until the cisterna becomes part of the \emph{trans}-Golgi network and subsequently fragments into secretory vesicles and other carriers\cite{glick_membrane_2009,glick_models_2011,pantazopoulou_kinetic_2019}. The cisternal maturation theory improves upon the thought that Golgi cisternae are stable entities with specific compositions, as this compartmentalized idea is not compatible with observations of the distribution of glycosyltransferases and other Golgi-associated proteins\cite{velasco_cell_1993,pantazopoulou_kinetic_2019,nilsson_kin_1993,rabouille_mapping_1995,harris_localization_1996,munro_what_2001}. The control of trafficking pathways to, from, and within the Golgi are also needed to dictate Golgi transport dynamics\cite{pantazopoulou_kinetic_2019} and important components include adapter proteins\cite{pantazopoulou_kinetic_2019}, GTPases\cite{mizuno-yamasaki_gtpase_2012}, tethering proteins\cite{witkos_golgin_2016,wong_membrane_2014} and SNARE proteins\cite{malsam_organization_2011}.

Not only the basal dynamics of the Golgi apparatus itself but also the kinetics of proteins that pass through the Golgi are of importance for the entire Golgi trafficking model. Considering two different types of secretory cargo, one transmembrane and the other soluble, large differences appear between their trafficking routes\cite{beznoussenko_transport_2014}. In this case, albumin, a soluble, non-glycosylated, protein that is secreted can traverse the Golgi within 5 minutes while a model transmembrane glycoprotein VSVG (vesicular stomatitis virus G protein) moves much slower through the Golgi\cite{beznoussenko_transport_2014}. Similarly, the transport of procollagen is also distinct as its large size (> 250 nm) is incompatible with the size of transport vesicles\cite{beznoussenko_transport_2014,mccaughey_er--golgi_2018}. These observations suggest that (glyco)protein topology and potential post-translational modifications influence the transport of proteins through the Golgi apparatus, meaning that separate transport routes for each protein must exist. 

Further insights into the role of glycosylation in the transit of proteins through the Golgi can be gathered by engineering a model protein with one or more glycosylation sites, potentially with different locations in the amino acid sequence, and observing how the trafficking of the protein is altered. Moreover, Golgi transport in different cell types can be fundamentally different, depending on the expression levels of trafficking machinery and Golgi-resident proteins. This is an important point as the glycosylation requirement of certain cell types (i.e., cell types highly dependent on secretion) can reshape intracellular membrane trafficking pathways. The opposite is also true: differences in intracellular membrane trafficking pathways can reshape glycosylation mechanisms in different cell types. Evidence for this also arises from this thesis, as I demonstrated in chapters 4 and 6 that modulating Golgi trafficking components (a putative factor in Golgi acidification in chapter 4, and SNARE protein syntaxin-5 in chapter 6) strongly affects the structure of the Golgi and the localization of glycosyltransferases with striking downstream effects on glycosylation. 

\section{Concluding Remarks}

Ultimately, the various insights gathered from advances in microscopy, Golgi transport, glycobiology and diagnostic strategies for CDGs together strengthen our knowledge and will benefit patients down the line. The results presented in this thesis identify the mechanism behind two distinct glycosylation disorders and thus offer the basis for novel therapeutic opportunities. Fundamentally understanding how both Golgi transport and glycosylation are linked to each other is therefore of utmost importance for the treatment of CDG.